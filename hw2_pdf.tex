% Options for packages loaded elsewhere
\PassOptionsToPackage{unicode}{hyperref}
\PassOptionsToPackage{hyphens}{url}
%
\documentclass[
]{article}
\title{HW 2}
\author{}
\date{\vspace{-2.5em}}

\usepackage{amsmath,amssymb}
\usepackage{lmodern}
\usepackage{iftex}
\ifPDFTeX
  \usepackage[T1]{fontenc}
  \usepackage[utf8]{inputenc}
  \usepackage{textcomp} % provide euro and other symbols
\else % if luatex or xetex
  \usepackage{unicode-math}
  \defaultfontfeatures{Scale=MatchLowercase}
  \defaultfontfeatures[\rmfamily]{Ligatures=TeX,Scale=1}
\fi
% Use upquote if available, for straight quotes in verbatim environments
\IfFileExists{upquote.sty}{\usepackage{upquote}}{}
\IfFileExists{microtype.sty}{% use microtype if available
  \usepackage[]{microtype}
  \UseMicrotypeSet[protrusion]{basicmath} % disable protrusion for tt fonts
}{}
\makeatletter
\@ifundefined{KOMAClassName}{% if non-KOMA class
  \IfFileExists{parskip.sty}{%
    \usepackage{parskip}
  }{% else
    \setlength{\parindent}{0pt}
    \setlength{\parskip}{6pt plus 2pt minus 1pt}}
}{% if KOMA class
  \KOMAoptions{parskip=half}}
\makeatother
\usepackage{xcolor}
\IfFileExists{xurl.sty}{\usepackage{xurl}}{} % add URL line breaks if available
\IfFileExists{bookmark.sty}{\usepackage{bookmark}}{\usepackage{hyperref}}
\hypersetup{
  pdftitle={HW 2},
  hidelinks,
  pdfcreator={LaTeX via pandoc}}
\urlstyle{same} % disable monospaced font for URLs
\usepackage[margin=1in]{geometry}
\usepackage{color}
\usepackage{fancyvrb}
\newcommand{\VerbBar}{|}
\newcommand{\VERB}{\Verb[commandchars=\\\{\}]}
\DefineVerbatimEnvironment{Highlighting}{Verbatim}{commandchars=\\\{\}}
% Add ',fontsize=\small' for more characters per line
\usepackage{framed}
\definecolor{shadecolor}{RGB}{248,248,248}
\newenvironment{Shaded}{\begin{snugshade}}{\end{snugshade}}
\newcommand{\AlertTok}[1]{\textcolor[rgb]{0.94,0.16,0.16}{#1}}
\newcommand{\AnnotationTok}[1]{\textcolor[rgb]{0.56,0.35,0.01}{\textbf{\textit{#1}}}}
\newcommand{\AttributeTok}[1]{\textcolor[rgb]{0.77,0.63,0.00}{#1}}
\newcommand{\BaseNTok}[1]{\textcolor[rgb]{0.00,0.00,0.81}{#1}}
\newcommand{\BuiltInTok}[1]{#1}
\newcommand{\CharTok}[1]{\textcolor[rgb]{0.31,0.60,0.02}{#1}}
\newcommand{\CommentTok}[1]{\textcolor[rgb]{0.56,0.35,0.01}{\textit{#1}}}
\newcommand{\CommentVarTok}[1]{\textcolor[rgb]{0.56,0.35,0.01}{\textbf{\textit{#1}}}}
\newcommand{\ConstantTok}[1]{\textcolor[rgb]{0.00,0.00,0.00}{#1}}
\newcommand{\ControlFlowTok}[1]{\textcolor[rgb]{0.13,0.29,0.53}{\textbf{#1}}}
\newcommand{\DataTypeTok}[1]{\textcolor[rgb]{0.13,0.29,0.53}{#1}}
\newcommand{\DecValTok}[1]{\textcolor[rgb]{0.00,0.00,0.81}{#1}}
\newcommand{\DocumentationTok}[1]{\textcolor[rgb]{0.56,0.35,0.01}{\textbf{\textit{#1}}}}
\newcommand{\ErrorTok}[1]{\textcolor[rgb]{0.64,0.00,0.00}{\textbf{#1}}}
\newcommand{\ExtensionTok}[1]{#1}
\newcommand{\FloatTok}[1]{\textcolor[rgb]{0.00,0.00,0.81}{#1}}
\newcommand{\FunctionTok}[1]{\textcolor[rgb]{0.00,0.00,0.00}{#1}}
\newcommand{\ImportTok}[1]{#1}
\newcommand{\InformationTok}[1]{\textcolor[rgb]{0.56,0.35,0.01}{\textbf{\textit{#1}}}}
\newcommand{\KeywordTok}[1]{\textcolor[rgb]{0.13,0.29,0.53}{\textbf{#1}}}
\newcommand{\NormalTok}[1]{#1}
\newcommand{\OperatorTok}[1]{\textcolor[rgb]{0.81,0.36,0.00}{\textbf{#1}}}
\newcommand{\OtherTok}[1]{\textcolor[rgb]{0.56,0.35,0.01}{#1}}
\newcommand{\PreprocessorTok}[1]{\textcolor[rgb]{0.56,0.35,0.01}{\textit{#1}}}
\newcommand{\RegionMarkerTok}[1]{#1}
\newcommand{\SpecialCharTok}[1]{\textcolor[rgb]{0.00,0.00,0.00}{#1}}
\newcommand{\SpecialStringTok}[1]{\textcolor[rgb]{0.31,0.60,0.02}{#1}}
\newcommand{\StringTok}[1]{\textcolor[rgb]{0.31,0.60,0.02}{#1}}
\newcommand{\VariableTok}[1]{\textcolor[rgb]{0.00,0.00,0.00}{#1}}
\newcommand{\VerbatimStringTok}[1]{\textcolor[rgb]{0.31,0.60,0.02}{#1}}
\newcommand{\WarningTok}[1]{\textcolor[rgb]{0.56,0.35,0.01}{\textbf{\textit{#1}}}}
\usepackage{graphicx}
\makeatletter
\def\maxwidth{\ifdim\Gin@nat@width>\linewidth\linewidth\else\Gin@nat@width\fi}
\def\maxheight{\ifdim\Gin@nat@height>\textheight\textheight\else\Gin@nat@height\fi}
\makeatother
% Scale images if necessary, so that they will not overflow the page
% margins by default, and it is still possible to overwrite the defaults
% using explicit options in \includegraphics[width, height, ...]{}
\setkeys{Gin}{width=\maxwidth,height=\maxheight,keepaspectratio}
% Set default figure placement to htbp
\makeatletter
\def\fps@figure{htbp}
\makeatother
\setlength{\emergencystretch}{3em} % prevent overfull lines
\providecommand{\tightlist}{%
  \setlength{\itemsep}{0pt}\setlength{\parskip}{0pt}}
\setcounter{secnumdepth}{-\maxdimen} % remove section numbering
\ifLuaTeX
  \usepackage{selnolig}  % disable illegal ligatures
\fi

\begin{document}
\maketitle

\hypertarget{question-8}{%
\section{Question 8}\label{question-8}}

\begin{Shaded}
\begin{Highlighting}[]
\FunctionTok{require}\NormalTok{(tidyverse)}
\end{Highlighting}
\end{Shaded}

\begin{verbatim}
## Loading required package: tidyverse
\end{verbatim}

\begin{verbatim}
## -- Attaching packages --------------------------------------- tidyverse 1.3.1 --
\end{verbatim}

\begin{verbatim}
## v ggplot2 3.3.5     v purrr   0.3.4
## v tibble  3.1.6     v dplyr   1.0.7
## v tidyr   1.1.4     v stringr 1.4.0
## v readr   2.1.1     v forcats 0.5.1
\end{verbatim}

\begin{verbatim}
## -- Conflicts ------------------------------------------ tidyverse_conflicts() --
## x dplyr::filter() masks stats::filter()
## x dplyr::lag()    masks stats::lag()
\end{verbatim}

\begin{Shaded}
\begin{Highlighting}[]
\FunctionTok{require}\NormalTok{(ISLR2)}
\end{Highlighting}
\end{Shaded}

\begin{verbatim}
## Loading required package: ISLR2
\end{verbatim}

\hypertarget{creating-lm}{%
\subparagraph{Creating lm}\label{creating-lm}}

\begin{Shaded}
\begin{Highlighting}[]
\NormalTok{df }\OtherTok{\textless{}{-}}\NormalTok{ Auto}

\NormalTok{model }\OtherTok{\textless{}{-}} \FunctionTok{lm}\NormalTok{(mpg }\SpecialCharTok{\textasciitilde{}}\NormalTok{ horsepower, }\AttributeTok{data =}\NormalTok{ df)}

\FunctionTok{summary}\NormalTok{(model)}
\end{Highlighting}
\end{Shaded}

\begin{verbatim}
## 
## Call:
## lm(formula = mpg ~ horsepower, data = df)
## 
## Residuals:
##      Min       1Q   Median       3Q      Max 
## -13.5710  -3.2592  -0.3435   2.7630  16.9240 
## 
## Coefficients:
##              Estimate Std. Error t value Pr(>|t|)    
## (Intercept) 39.935861   0.717499   55.66   <2e-16 ***
## horsepower  -0.157845   0.006446  -24.49   <2e-16 ***
## ---
## Signif. codes:  0 '***' 0.001 '**' 0.01 '*' 0.05 '.' 0.1 ' ' 1
## 
## Residual standard error: 4.906 on 390 degrees of freedom
## Multiple R-squared:  0.6059, Adjusted R-squared:  0.6049 
## F-statistic: 599.7 on 1 and 390 DF,  p-value: < 2.2e-16
\end{verbatim}

There is a clear relationship between \texttt{horsepower} and
\texttt{mpg} in the \texttt{Auto} dataset. The \texttt{horsepower}
coefficient of \texttt{-0.16} indicates that for every one additional
unit of \texttt{horsepower} corresponds with less \texttt{mpg}. The
negative relationship is fairly strong, though it is interesting that
it's more of a non-linear relationship at higher \texttt{horsepower}
values, where the difference between 150 units of \texttt{horsepower}
and 200 units of \texttt{horsepower} is fairly small, whereas the
difference between 75 units and 100 units is far more pronounced.

\begin{Shaded}
\begin{Highlighting}[]
\NormalTok{new }\OtherTok{\textless{}{-}} \FunctionTok{data.frame}\NormalTok{(}\AttributeTok{horsepower =} \FunctionTok{c}\NormalTok{(}\DecValTok{98}\NormalTok{))}

\NormalTok{pred }\OtherTok{\textless{}{-}} \FunctionTok{predict}\NormalTok{(model, }\AttributeTok{newdata =}\NormalTok{ new, }\AttributeTok{interval =} \StringTok{"confidence"}\NormalTok{) }\SpecialCharTok{\%\textgreater{}\%}
  \FunctionTok{data.frame}\NormalTok{()}
\end{Highlighting}
\end{Shaded}

The predicted \texttt{mpg} of a 98 \texttt{horsepower} is 24.4670772,
while the confidence interval ranges from 23.973079 to 24.9610753. Below
is a plot of \texttt{mpg} as a function of \texttt{horsepower}, with the
least squares regression line plotted over it.

\begin{Shaded}
\begin{Highlighting}[]
\CommentTok{\# df \%\textgreater{}\%}
\CommentTok{\#   ggplot(aes(x = horsepower, y = mpg)) +}
\CommentTok{\#   geom\_point() +}
\CommentTok{\#   geom\_smooth(method = "lm", formula = mpg \textasciitilde{} horsepower) +}
\CommentTok{\#   theme\_minimal() + }
\CommentTok{\#   labs(title = "Horsepower vs MPG") +}
\CommentTok{\#   theme(plot.title = element\_text(hjust = 0.5, size = 14, face = "bold"))}

\FunctionTok{with}\NormalTok{(df, }\FunctionTok{plot}\NormalTok{(horsepower, mpg)) }\SpecialCharTok{+}
\FunctionTok{abline}\NormalTok{(model)}
\end{Highlighting}
\end{Shaded}

\begin{flushleft}\includegraphics{hw2_pdf_files/figure-latex/unnamed-chunk-4-1} \end{flushleft}

\begin{verbatim}
## integer(0)
\end{verbatim}

\begin{Shaded}
\begin{Highlighting}[]
\FunctionTok{plot}\NormalTok{(model)}
\end{Highlighting}
\end{Shaded}

\begin{flushleft}\includegraphics{hw2_pdf_files/figure-latex/unnamed-chunk-5-1} \end{flushleft}

\begin{flushleft}\includegraphics{hw2_pdf_files/figure-latex/unnamed-chunk-5-2} \end{flushleft}

\begin{flushleft}\includegraphics{hw2_pdf_files/figure-latex/unnamed-chunk-5-3} \end{flushleft}

\begin{flushleft}\includegraphics{hw2_pdf_files/figure-latex/unnamed-chunk-5-4} \end{flushleft}

From evaluating the diagnostic plots, it's clear that the linear model
is not a good fit for the current data set. The residuals are not evenly
or consistently distributed over the distribution of fitted values. For
example, the model consistently under predicts low values of
\texttt{mpg}. The large positive residual shows that the actual values
are well above the prediction. A similar phenomenon occurs at higher
values too; given the \texttt{horsepower\ vs\ mpg} and line of best fit
plot, the non-linear relationship between the two variables ends up
under-predicting values at both extremes.

\end{document}
